%%
%% This is file `mcmthesis-demo.tex',
%% generated with the docstrip utility.
%%
%% The original source files were:
%%
%% mcmthesis.dtx  (with options: `demo')
%% !Mode:: "TeX:UTF-8"
%% -----------------------------------
%%
%% This is a generated file.
%%
%% Copyright (C)
%%     2010 -- 2015 by latexstudio
%%     2014 -- 2016 by Liam Huang
%%
%% This work may be distributed and/or modified under the
%% conditions of the LaTeX Project Public License, either version 1.3
%% of this license or (at your option) any later version.
%% The latest version of this license is in
%%   http://www.latex-project.org/lppl.txt
%% and version 1.3 or later is part of all distributions of LaTeX
%% version 2005/12/01 or later.
%%
%% This work has the LPPL maintenance status `maintained'.
%%
%% The Current Maintainer of this work is Liam Huang.
%%
\documentclass{mcmthesis}
\mcmsetup{CTeX = false,   % 使用 CTeX 套装时,设置为 true
        tcn = 78460, problem = C,
        sheet = true, titleinsheet = true, keywordsinsheet = true,
        titlepage = true, abstract = true}
\usepackage{palatino}
\usepackage{lipsum}
\usepackage{float}
\title{General assessment of green energy development in Arizona, California, New Mexico and Texas}

\def\az{Arizona}
\def\ca{California}
\def\nm{New Mexico}
\def\tx{Texas}

\begin{document}
\begin{abstract}

\begin{keywords}
keyword1; keyword2
\end{keywords}
\end{abstract}
\maketitle


\section{Introduction}
We tackle $ 4 $ sub-problems:
\begin{itemize}[-]
\item \textbf{to find quantitative relationships between different energy sources}
\item \textbf{how to analyse four states energy structure and predict the future energy trends based on 1960 to 2009 data}
\item \textbf{how to quantify the "cleanness index" of four state energy structure}
\item \textbf{how to quantify and predict every factors impact on total renewable or non-renewable (fossil fuel) energy}
\end{itemize}


At first, we statistically analyse four state energy structure based on electricity, coal, LPG, and natural gas production and/or consumption structure. To find the association of different sectors consumption (transportation, commerce, residence, and industry etc.) with total consumption, we calculate the correlation coefficient and overall percentage for each sectors respectively. 
To address the second sub-problem, we use different regression models to fit the state-wide energy consumption and production structure over decades. Every regression models are calculated based on 50 years and 20 years data separately, and we will illustrate each model mathematical significance and assumptions. We project 2025 and 2050 energy profile based on our regression models while discussing each regression model implications potential weakness.\\

Next, we develop a hierarchical database model to define overall "cleanness index" based on each state energy production and consumption structure. We will quantify each energy source "cleanness" and develop an overall model to discuss the "cleanest" state.\\

Finally, we will associate other implicit factors, such as population growth and industrial development, with the changes in renewable energy and fossil fuel energy structure. We will recommend several policies for four state governors to reduce each state fossil fuels uses.


\subsection{Assumptions}
Our general assumptions are listed below. Additional assumptions may be added when performing further analysis.

\begin{enumerate}[1]
\item \textbf{The policy scenario for renewable energy and fossil fuels will not change from 2009 to 2050.}
\item \textbf{The technology of refining fossil fuels, exploit wind, solar, hydro, geothermal energy, transporting the electricity, natural gas, LPG will not achieve a breakthrough in the near future, particularly in 2050.}
\item \textbf{The mathematical significance of each regression model will be valid in predicting future energy trend. }
\item \textbf{Arizona, New Mexico, California, Texas regional climate, geography, and employable natural resources will be stable in the near future. That is, all carrying capacity for natural resources in 2050 will be essentially the same as in 2009. }
\item \textbf{All implicit factors are respectively independent.}
\end{enumerate}

\section{Definition of Concepts and Variables}

\section{Part A: Analysis and Prediction of State Energy Profiles}
\subsection{Energy Profile of Each State in 2009}
	\subsubsection{\az}
		General Energy profile for \az in year 2009 shown below in \ref{tab:data}\\
		\begin{table}[H]
  		\centering
			\begin{tabular}{c|c}
			Name of Parameter (units) & Value\\ \hline
			Total Population in \az (thousands) & \\
			Total Energy Consumption (Billion Btu) & \\ %\hline
			\end{tabular}
			\caption{General Energy Profile}
			\label{tab:data}
		 \end{table}
	\subsubsection{\ca}
		The energy profile for \ca as of year 2009 is shown below
	\subsubsection{\nm}
		The energy profile for \nm as of year 2009 is shown below
	\subsubsection{\tx}
		The energy profile for \tx as of year 2009 is shown below
		
\subsection{Evaluation of Energy Profile}
	\subsubsection{Evaluation Method}
		why, how, what
	\subsubsection{Analysis and testing of our method}
		historical data

\subsection{Models for Energy Profile Characteristics(1960~2009) and prediction(2025~2050)}
	methods of testing and models of fitting -> discuss here
	\subsubsection{General Trend of Energy profile}
		Usage of fossil feul, growth rate of fossil feul usage
		total, growth rate, 
		electricity, growth rate
		Renewable energy in total , growth rate			
	\subsubsection{Analysis of Major Renewable Energy Source}
		Wind
		Water
		Geothermal
		Biomass
		Solar
	\subsubsection{Analysis of Consumer Sectors and Influential Factors}	
		Hierarchy of Influence
		Transportation
		Residential
		Industrial
		Commercial
		
		Population, GDP(economic factor), General climate data, resources
	\subsubsection{Financial Burden Analysis}
		ratio of renewable energy in total expenditure
		average price vs. non-renewable
		table of average price
	\subsubsection{Prediction}
		result ->
	\subsubsection{Analysis and testing of our models of prediction}
		recent data ->
		
\section{Part B: Goals and Policies}
\subsection{Setting Renewable Energy Usage Targets}
	Base on evaluation model and current profile and prediction
	
\subsection{Policy Suggestions}
	Change how things develop as predicted

\section{Conclusions}

\section{A Summary}

\section{Strengths and weaknesses}

\subsection{Strengths}
\subsection{Weaknesses}

\section{A Letter to the Governors}


\begin{thebibliography}{99}
\bibitem{1} D.~E. KNUTH   The \TeX{}book  the American
Mathematical Society and Addison-Wesley
Publishing Company , 1984-1986.
\bibitem{2}Lamport, Leslie,  \LaTeX{}: `` A Document Preparation System '',
Addison-Wesley Publishing Company, 1986.
\bibitem{3}\url{http://www.latexstudio.net/}
\bibitem{4}\url{http://www.chinatex.org/}
\end{thebibliography}

\begin{appendices}

\section{First appendix}

\lipsum[13]

Here are simulation programmes we used in our model as follow.\\

\textbf{\textcolor[rgb]{0.98,0.00,0.00}{Input matlab source:}}
\lstinputlisting[language=Matlab]{./code/mcmthesis-matlab1.m}

\section{Second appendix}

some more text \textcolor[rgb]{0.98,0.00,0.00}{\textbf{Input C++ source:}}
\lstinputlisting[language=C++]{./code/mcmthesis-sudoku.cpp}

\end{appendices}
\end{document}

%%
%% This work consists of these files mcmthesis.dtx,
%%                                   figures/ and
%%                                   code/,
%% and the derived files             mcmthesis.cls,
%%                                   mcmthesis-demo.tex,
%%                                   README,
%%                                   LICENSE,
%%                                   mcmthesis.pdf and
%%                                   mcmthesis-demo.pdf.
%%
%% End of file `mcmthesis-demo.tex'.
